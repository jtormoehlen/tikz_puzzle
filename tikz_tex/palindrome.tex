\begin{tikzpicture}

    \node[draw, rounded corners, fill=lightgray!20, minimum width=0.5*\rectWidth, minimum height=\rectHeight, anchor=north west] (rect1) at (0,0) {
\begin{minipage}[c]{6.5cm}
\begin{lstlisting}[language=Python, breaklines=true]
subseq = dna[start:start + length] 
\end{lstlisting}
\end{minipage}
\begin{minipage}[c]{0.5cm}
0b
\end{minipage}
};\node[draw, rounded corners, fill=lightgray!20, minimum width=0.5*\rectWidth, minimum height=\rectHeight, anchor=west] (rect101) at (rect1.east) {
\begin{minipage}[c]{6.5cm}
\begin{lstlisting}[language=Python, breaklines=true]
 subseq = dna[start:length]
\end{lstlisting}
\end{minipage}
\begin{minipage}[c]{0.5cm}
0c
\end{minipage}
};\node[draw, rounded corners, fill=lightgray!20, minimum width=1*\rectWidth, minimum height=\rectHeight, anchor=north west] (rect2) at (rect1.south west) {
\begin{minipage}[c]{14.5cm}
\begin{lstlisting}[language=Python, breaklines=true]
if is_palindromic(subseq):
\end{lstlisting}
\end{minipage}
\begin{minipage}[c]{0.5cm}
1
\end{minipage}
};\node[draw, rounded corners, fill=lightgray!20, minimum width=0.5*\rectWidth, minimum height=\rectHeight, anchor=north west] (rect3) at (rect2.south west) {
\begin{minipage}[c]{6.5cm}
\begin{lstlisting}[language=Python, breaklines=true]
return seq == get_compl(seq[::-1]) 
\end{lstlisting}
\end{minipage}
\begin{minipage}[c]{0.5cm}
2b
\end{minipage}
};\node[draw, rounded corners, fill=lightgray!20, minimum width=0.5*\rectWidth, minimum height=\rectHeight, anchor=west] (rect301) at (rect3.east) {
\begin{minipage}[c]{6.5cm}
\begin{lstlisting}[language=Python, breaklines=true]
 return seq == seq[::-1]
\end{lstlisting}
\end{minipage}
\begin{minipage}[c]{0.5cm}
2c
\end{minipage}
};\node[draw, rounded corners, fill=lightgray!20, minimum width=0.5*\rectWidth, minimum height=\rectHeight, anchor=north west] (rect4) at (rect3.south west) {
\begin{minipage}[c]{6.5cm}
\begin{lstlisting}[language=Python, breaklines=true]
if start >= end: 
\end{lstlisting}
\end{minipage}
\begin{minipage}[c]{0.5cm}
3b
\end{minipage}
};\node[draw, rounded corners, fill=lightgray!20, minimum width=0.5*\rectWidth, minimum height=\rectHeight, anchor=west] (rect401) at (rect4.east) {
\begin{minipage}[c]{6.5cm}
\begin{lstlisting}[language=Python, breaklines=true]
 if start < end:
\end{lstlisting}
\end{minipage}
\begin{minipage}[c]{0.5cm}
3c
\end{minipage}
};\node[draw, rounded corners, fill=lightgray!20, minimum width=1*\rectWidth, minimum height=\rectHeight, anchor=north west] (rect5) at (rect4.south west) {
\begin{minipage}[c]{14.5cm}
\begin{lstlisting}[language=Python, breaklines=true]
def longest_palindrome(dna, start, end, longest_seq=''):
\end{lstlisting}
\end{minipage}
\begin{minipage}[c]{0.5cm}
4
\end{minipage}
};\node[draw, rounded corners, fill=lightgray!20, minimum width=0.5*\rectWidth, minimum height=\rectHeight, anchor=north west] (rect6) at (rect5.south west) {
\begin{minipage}[c]{6.5cm}
\begin{lstlisting}[language=Python, breaklines=true]
return longest_palindrome(dna, start + 1, end, longest_seq) 
\end{lstlisting}
\end{minipage}
\begin{minipage}[c]{0.5cm}
5b
\end{minipage}
};\node[draw, rounded corners, fill=lightgray!20, minimum width=0.5*\rectWidth, minimum height=\rectHeight, anchor=west] (rect601) at (rect6.east) {
\begin{minipage}[c]{6.5cm}
\begin{lstlisting}[language=Python, breaklines=true]
 return longest_palindrome(dna, start + 1, end - 1, longest_seq)
\end{lstlisting}
\end{minipage}
\begin{minipage}[c]{0.5cm}
5c
\end{minipage}
};\node[draw, rounded corners, fill=lightgray!20, minimum width=1*\rectWidth, minimum height=\rectHeight, anchor=north west] (rect7) at (rect6.south west) {
\begin{minipage}[c]{14.5cm}
\begin{lstlisting}[language=Python, breaklines=true]
for length in range(2, end - start + 1):
\end{lstlisting}
\end{minipage}
\begin{minipage}[c]{0.5cm}
6
\end{minipage}
};\node[draw, rounded corners, fill=lightgray!20, minimum width=0.5*\rectWidth, minimum height=\rectHeight, anchor=north west] (rect8) at (rect7.south west) {
\begin{minipage}[c]{6.5cm}
\begin{lstlisting}[language=Python, breaklines=true]
longest_seq = subseq 
\end{lstlisting}
\end{minipage}
\begin{minipage}[c]{0.5cm}
7b
\end{minipage}
};\node[draw, rounded corners, fill=lightgray!20, minimum width=0.5*\rectWidth, minimum height=\rectHeight, anchor=west] (rect801) at (rect8.east) {
\begin{minipage}[c]{6.5cm}
\begin{lstlisting}[language=Python, breaklines=true]
 return subseq
\end{lstlisting}
\end{minipage}
\begin{minipage}[c]{0.5cm}
7c
\end{minipage}
};\node[draw, rounded corners, fill=lightgray!20, minimum width=1*\rectWidth, minimum height=\rectHeight, anchor=north west] (rect9) at (rect8.south west) {
\begin{minipage}[c]{14.5cm}
\begin{lstlisting}[language=Python, breaklines=true]
if len(subseq) > len(longest_seq):
\end{lstlisting}
\end{minipage}
\begin{minipage}[c]{0.5cm}
8
\end{minipage}
};\node[draw, rounded corners, fill=lightgray!20, minimum width=0.5*\rectWidth, minimum height=\rectHeight, anchor=north west] (rect10) at (rect9.south west) {
\begin{minipage}[c]{6.5cm}
\begin{lstlisting}[language=Python, breaklines=true]
return longest_seq 
\end{lstlisting}
\end{minipage}
\begin{minipage}[c]{0.5cm}
9b
\end{minipage}
};\node[draw, rounded corners, fill=lightgray!20, minimum width=0.5*\rectWidth, minimum height=\rectHeight, anchor=west] (rect1001) at (rect10.east) {
\begin{minipage}[c]{6.5cm}
\begin{lstlisting}[language=Python, breaklines=true]
 return ''
\end{lstlisting}
\end{minipage}
\begin{minipage}[c]{0.5cm}
9c
\end{minipage}
};\node[draw, rounded corners, fill=lightgray!20, minimum width=1*\rectWidth, minimum height=\rectHeight, anchor=north west] (rect11) at (rect10.south west) {
\begin{minipage}[c]{14.5cm}
\begin{lstlisting}[language=Python, breaklines=true]
def is_palindromic(seq):
\end{lstlisting}
\end{minipage}
\begin{minipage}[c]{0.5cm}
10
\end{minipage}
};

    \end{tikzpicture}
